\documentclass[12pt]{article}
\usepackage[utf8]{inputenc}
\usepackage[T1]{fontenc}
\usepackage{amsmath}
\usepackage{amsfonts}
\usepackage{amssymb}
\usepackage{graphicx}
\usepackage{geometry}
\usepackage{setspace}
\usepackage{natbib}
\usepackage{hyperref}
\usepackage{amsmath}
\usepackage{amsfonts}
\usepackage{amssymb}
\usepackage{geometry}
\usepackage{setspace}
\usepackage{float}
\usepackage{booktabs}
\usepackage{rotating}
\usepackage{amsmath}
\usepackage{amssymb}


\geometry{a4paper, margin=1in}

\geometry{margin=1in}
\doublespacing

\title{Problem Set 5}
\author{Cem Kozanoglu}

\date{\today}

\begin{document}

\maketitle

Collaborators: Wooyong Park, Roberto Gonzalez-Tellez, Hanniel Ho and Aileen Wu.

\section{Problem 1}

\subsection{Part A}

\subsubsection{Model}

We estimate the following model:

$$ \text{lwage}_{i} = \beta_0 + \beta_1 \text{educ}_{i} + \beta_2 \text{state\_avg\_educ}_{i} + \epsilon_i $$

Where $\text{lwage}_{i}$ is the log of wage for individual $i$, $\text{educ}_{i}$ is the years of education for individual $i$, and $\text{state\_avg\_educ}_{i}$ is the average years of education in the state where individual $i$ resides. The error term $\epsilon_i$ captures unobserved factors affecting wages.

\subsubsection{Regression Results}
\begin{table}[H]
\centering
\begin{tabular}{lc}
\toprule
 & (1) \\
\midrule
const & 4.9952*** \\
 & (0.004) \\
educ & 0.0709*** \\
 & (0.000) \\
\midrule
R-squared & 0.117 \\
Observations & 329,509 \\
\bottomrule
\end{tabular}
\caption{OLS Regression Results}
\label{tab:ols_results}
\end{table}

These results show that the coefficient on education is positive and statistically significant at the 1\% level. This suggests that, on average, an additional year of education is associated with a 7.09\% increase in wages. The R-squared value of 0.117 indicates that approximately 11.7\% of the variation in log wages can be explained by years of education alone.

\subsection{Part B, C}

\subsubsection{Standard Error Formulas}

The formula for the Homoskedastic OLS standard error is:

$$ SE(\hat{\beta}) = \sqrt{\frac{1}{N} \cdot \frac{\sum (\epsilon_i - \bar{\epsilon})^2}{\sum (X_i - \bar{X})^2}} $$

Where $\sigma^2$ is the variance of the error term, and $\sum (X_i - \bar{X})^2$ is the sum of squared deviations of the independent variable from its mean.

Whereas the formula for the clustered standard error is:

$$\text{SE}(\hat{\beta}_j) = \sqrt{  (X'X)^{-1} \left( \sum_{c} X_c' \hat{\Omega}_c X_c \right) (X'X)^{-1} \frac{C}{C-1} \frac{n-1}{n-k}  }$$

Where $C$ is the number of clusters, $n$ is the total number of observations, and $k$ is the number of parameters estimated. The term $\hat{\Omega}_c$ represents the covariance matrix of the residuals within cluster $c$.

\subsubsection{Empirical results}

\begin{table}[H]
\centering
\begin{tabular}{lc}
\toprule
Standard Error Type & Value \\
\midrule
Homoskedastic & 0.0003386 \\
Clustered & 0.0018 \\
\bottomrule
\end{tabular}
\caption{Standard Error Comparison}
\label{tab:se_comparison}
\end{table}

As we can see from the table, the clustered standard error is significantly larger than the homoskedastic standard error. This suggests that there is likely some correlation of errors within clusters (states), which violates the homoskedasticity assumption and leads to underestimated standard errors when using the homoskedastic formula. Despite this, we still observe that the coefficient on education is statistically significant at the 1\% level, even when using the larger clustered standard error.

\subsection{Part D}

\subsubsection{Bootstrap algorithms}

For the OLS standard error, I run a simple Parametric bootstrap by resampling the residuals from the OLS regression and recalculating the OLS estimates for each bootstrap sample.

For the clustered standard error, I first run a Clustered bootstrap by resampling clusters (states) with replacement and then sample observations within these clusters to create our full bootstrap sample. I calculate the coefficients for each bootstrap sample and compute the standard error across these bootstrap estimates.

\subsubsection{Empirical Results}

The empirical results from the bootstrap methods after 10,000 simulations for both are as follows:

\begin{table}[H]
\centering
\begin{tabular}{lc}
\toprule
Bootstrap Method & Standard Error \\
\midrule
OLS & 0.0003356 \\
Clustered & 0.0017811 \\
\bottomrule
\end{tabular}
\caption{Bootstrap Standard Error Results}
\label{tab:bootstrap_se}
\end{table}

As we can see, the values are very close to the standard errors calculated using the formulas, which provides validation for our bootstrap approach.

\subsection{Part E}

\subsubsection{Model}

In this section we calculate a slightly different model by including the state average education as an additional regressor. The model is as follows:

$$ \text{lwage}_{i} = \beta_0 + \beta_1 \text{educ}_{i} + \beta_2 \text{state\_avg\_educ}_{i} + \epsilon_i $$

Where $\text{lwage}_{i}$ is the log of wage for individual $i$, $\text{educ}_{i}$ is the years of education for individual $i$, and $\text{state\_avg\_educ}_{i}$ is the average years of education in the state where individual $i$ resides. The error term $\epsilon_i$ captures unobserved factors affecting wages.

The economic reason that we might want to include the average state education is that it could capture the overall educational environment of the state, which may have an impact on individual wages. For example, living in a state with higher average education levels might provide better job opportunities, networking effects, or a more skilled workforce, all of which could positively influence an individual's wage. In our case, we don't know how good a proxy state of birth is for state of residence, but we can assume that it is a reasonable proxy for the sake of this analysis. Even if it isn't it probably is a good proxy for state where the individual was educated, where we might further expect better educated states to have better education systems in general, so we have good reason to expect that one or both of these channels would be affecting an individual's wages.

\subsubsection{Regression Resuls}

We get the following results:

\begin{table}[H]
\centering
\begin{tabular}{lc}
\toprule
 & (1) \\
\midrule
const & $4.1542^{***}$ \\
 & $(0.019)$ \\
educ & $0.0671^{***}$ \\
 & $(0.000)$ \\
state\_avg\_educ & $0.0696^{***}$ \\
 & $(0.001)$ \\
\midrule
R-squared & $0.123$ \\
Observations & $329,509$ \\
\bottomrule
\end{tabular}
\caption{OLS Regression Results with State Average Education}
\label{tab:ols_results2}
\end{table}

We observe that all three coefficients are statistically significant at the 1\% level. The coefficient on education is slightly smaller than in the previous model, which suggests that some of the effect of education on wages is captured by the state average education variable. The coefficient on state average education is positive and significant, indicating that living in a state with higher average education levels is associated with higher wages. Interestingly though, the R-squared only shows a marginal increase, meaning that the state average education is not actually giving us that much more new information.

\subsection{Part F, G, H}

\subsubsection{Standard Error Results}

Using the same general formulas for the standard errors as in Part B and C, we get the following results:

\begin{table}[H]
\centering
\begin{tabular}{lcc}
\toprule
Standard Error Type & educ & state\_avg\_educ \\
\midrule
Homoskedastic & 0.0003375 & 0.0014547 \\
Clustered & 0.0013508 & 0.0119502 \\
\bottomrule
\end{tabular}
\caption{Standard Error Comparison for Part E}
\label{tab:se_comparison_partE}
\end{table}

I used the same bootstrap algorithms as in Part D to calculate the standard errors for the new model. The results are as follows after 10,000 simulations:

\begin{table}[H]
\centering
\begin{tabular}{lccc}
\toprule
Standard error type & Intercept & Education & State Avg Education \\
\midrule
OLS & 0.0186 & 0.0003 & 0.0015 \\
Clustered & 0.1612 & 0.0014 & 0.0122 \\
\bottomrule
\end{tabular}
\caption{Bootstrap Standard Errors for Part E}
\label{tab:bootstrap_se_partE}
\end{table}

We see quick convergence to the observed standard errors again, giving extra confidence in our standard error estimates.

Here we can make two observations regarding clustering. First, the clustered standard error for the state average education variable is much larger than the homoskedastic standard error, which suggests that there is significant within-cluster correlation for this variable. This makes sense because individuals within the same state are likely to share similar unobserved characteristics that affect their wages, leading to correlated errors.

An additional economic observations is that while the OLS standard error of the educ variable has not changed much from the first mode, the clustered standard error has decreased significantly. This is likely because the state average education variable captures some of the within-state variation in education levels and also some of the unobserved state-level factors, which reduces the correlation of errors within clusters for the educ variable. In other words, by including state average education, we are controlling for some of the unobserved factors that were previously causing correlated errors within states, leading to a more accurate estimate of the standard error for the educ variable.

\section{Problem 2}


\noindent We define the error covariance structure $\Omega$ as:
$$
\Omega_{ij} = \begin{cases} 
\sigma_{\eta}^2 + \sigma_{\nu}^2 & \text{if } i=j \\
\sigma_{\eta}^2 & \text{if } i \neq j, \ S_i = S_j \\
0 & \text{otherwise}
\end{cases}
$$
Then we can rewrite $\Omega$ in matrix notation as:
$$ \Omega = \sigma_{\nu}^2 I_N + \sigma_{\eta}^2 ZZ^\top $$

\noindent Furthermore, we can use the properties of $ZZ^\top$ and $X$ to get the following:
$$
\begin{aligned}
\Omega X &= (\sigma_{\nu}^2 I_N + \sigma_{\eta}^2 ZZ^\top) X \\
         &= \sigma_{\nu}^2 X + \sigma_{\eta}^2 ZZ^\top X
\end{aligned}
$$

\noindent \textbf{Analysis of the term $ZZ^\top X$:}
Since $X$ is constant within clusters, let $x_c$ be the row vector of regressor values for cluster $c$. We can visualize the multiplication of the block-diagonal matrix $ZZ^\top$ and $X$ as follows:

$$
ZZ^\top X = 
\begin{bmatrix} 
J_{N_1} & 0 & \cdots \\
0 & J_{N_2} & \cdots \\
\vdots & \vdots & \ddots
\end{bmatrix}
\begin{bmatrix} 
\mathbf{1}_{N_1} x_1 \\
\mathbf{1}_{N_2} x_2 \\
\vdots
\end{bmatrix}
$$

\noindent Where $J_{N_c}$ is an $N_c \times N_c$ matrix of ones, and $\mathbf{1}_{N_c}$ is a column vector of ones. For any specific cluster $c$:
$$
J_{N_c} (\mathbf{1}_{N_c} x_c) = 
\begin{bmatrix} 
1 & \cdots & 1 \\
\vdots & \ddots & \vdots \\
1 & \cdots & 1
\end{bmatrix}
\begin{bmatrix} 
x_c \\
\vdots \\
x_c
\end{bmatrix}
=
\begin{bmatrix} 
N_c x_c \\
\vdots \\
N_c x_c
\end{bmatrix}
= N_c (\mathbf{1}_{N_c} x_c)
$$

\noindent Assuming $N_c = N/C$ is constant across all clusters:
$$ ZZ^\top X = N_c X $$

\noindent Substituting this back into the expression for $\Omega X$:
$$
\begin{aligned}
\Omega X &= \sigma_{\nu}^2 X + \sigma_{\eta}^2 (N_c) X \\
\Omega X &= (\sigma_{\nu}^2 + N_c \sigma_{\eta}^2) X
\end{aligned}
$$
Let $\lambda = (\sigma_{\nu}^2 + N_c \sigma_{\eta}^2)$. Thus, $X$ is an eigenvector of $\Omega$ with eigenvalue $\lambda$.

\noindent \textbf{Calculating the Variance of $\hat{\beta}_{OLS}$:}
$$
\begin{aligned}
V(\hat{\beta}_{OLS}) &= (X^\top X)^{-1} (X^\top \Omega X) (X^\top X)^{-1} \\
&= (X^\top X)^{-1} X^\top (\lambda X) (X^\top X)^{-1} \\
&= \lambda (X^\top X)^{-1} (X^\top X) (X^\top X)^{-1} \\
&= (\sigma_{\nu}^2 + N_c \sigma_{\eta}^2)(X^\top X)^{-1}
\end{aligned}
$$

\noindent \textbf{Calculating the Variance of $\hat{\beta}_{GLS}$:}
Since $\Omega X = \lambda X$, it implies $\Omega^{-1} X = \frac{1}{\lambda} X$.
$$
\begin{aligned}
V(\hat{\beta}_{GLS}) &= (X^\top \Omega^{-1} X)^{-1} \\
&= \left( X^\top \frac{1}{\lambda} X \right)^{-1} \\
&= \lambda (X^\top X)^{-1} \\
&= (\sigma_{\nu}^2 + N_c \sigma_{\eta}^2)(X^\top X)^{-1}
\end{aligned}
$$

\noindent \textbf{Conclusion:}
$$ V(\hat{\beta}_{OLS}) = V(\hat{\beta}_{GLS}) = (\sigma_{\nu}^2 + N_c \sigma_{\eta}^2)(X^\top X)^{-1} $$

Therefore we've shown that the variance of the OLS estimator is equal to the variance of the GLS estimator under the given error covariance structure.

\subsection{Relevance}

In the slides, we saw that the variance of the OLS estimator is greater than or equal to the variance of the GLS estimator. In this problem, we've shown one case, when the regressors are constant within clusters, where the variance of the OLS estimator is actually equal to the variance of the GLS estimator. This illustrates that while GLS can be more efficient than OLS in general, there are specific cases where OLS can perform just as well as GLS, particularly when the regressors do not vary within clusters. This is an important insight because it highlights that the efficiency gains from using GLS depend on the structure of the regressors and the error covariance. 
\end{document}